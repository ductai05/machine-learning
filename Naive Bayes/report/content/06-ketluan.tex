% \newpage
\section{Đánh giá, kết luận}

\paragraph{}{Qua quá trình thực hiện báo cáo \textit{Lab 3 - Phân loại thư rác}, nhóm đã thành công trong việc tìm hiểu và tự lập trình xây dựng một mô hình \textit{Naive Bayes Classifier}. Mô hình được xây dựng dựa trên các kiến thức nền tảng về lý thuyết xác suất, \textit{Định lý Bayes}, cùng với các kỹ thuật ước lượng tham số như \textit{Maximum Likelihood Estimation (MLE)} cho xác suất tiên nghiệm của lớp và \textit{Maximum A Posteriori (MAP) Estimation} (thông qua \textit{Laplace Smoothing}) cho xác suất có điều kiện của từ. Nguyên tắc \textit{Bag-of-Words} đã được áp dụng hiệu quả để biểu diễn dữ liệu văn bản, tập trung vào tần suất xuất hiện của từ mà bỏ qua cấu trúc ngữ pháp phức tạp.}

\paragraph{}{Khi được huấn luyện và đánh giá trên bộ dữ liệu \textit{Enron-Spam}, mô hình của nhóm đã đạt được độ chính xác tổng thể trên 98.9\% trên tập huấn luyện và 98.5\% trên tập kiểm định, cho thấy khả năng học tốt các đặc điểm của dữ liệu và khả năng tổng quát hóa cao trên dữ liệu mới chưa từng thấy. Các chỉ số đánh giá chi tiết như Precision, Recall, và F1-Score cho cả hai lớp \texttt{``ham''} và \texttt{``spam''} đều ở mức rất cao (trên 0.98), phản ánh sự cân bằng và hiệu quả của mô hình trong việc vừa phát hiện chính xác thư rác, vừa hạn chế tối đa việc phân loại nhầm thư hợp lệ. Ma trận nhầm lẫn cũng cho thấy số lượng False Positives (23 trên tập validation) và False Negatives (26 trên tập validation) là tương đối thấp, khẳng định tính thực tiễn của giải pháp.}

\paragraph{}{Việc tự lập trình mô hình \textit{Naive Bayes Classifier} đã giúp nhóm hiểu sâu hơn và có thể áp dụng thực tiễn kiến thức nền tảng về lý thuyết xác suất, cùng các kỹ thuật MLE, MAP được học ở trên lớp để đảm bảo xây dựng mô hình hoạt động ổn định và hiệu quả.}